\documentclass[a4paper, 12pt, twoside, BCOR=20mm, DIV=calc, abstracton, parskip=half*, toc=bibliography, toc=listof, headsepline, headings=small, numbers=enddot]{scrreprt} %, footsepline
\usepackage{fancyhdr}
\pagestyle{fancy}					%Fußnotenformatierung ÃŒbernehmen
\fancyhf{}
\fancyheadoffset[RO,LE]{30pt}
\fancyhead{}
\fancyhead[RO,LE]{\thepage}
\fancyhead[LO]{\rightmark}
\fancyhead[RE]{\leftmark}
\fancyfoot{\title}
\renewcommand\headrulewidth{0.4pt}
\renewcommand\footrulewidth{0.4pt}
\usepackage[sorting=nty,language=ngerman,style=numeric]{biblatex}%bibencoding=latin1, backend=bibtex8, 
\addbibresource{literatur}
\usepackage[utf8]{inputenc}
\usepackage[T1]{fontenc}
\usepackage{selinput}
\usepackage[babel, german=quotes]{csquotes}
\usepackage[ngerman, english]{babel}
%\usepackage[automark]{scrpage2}	%Fußnoten formatieren


\usepackage{graphicx}						%Einbinden von Grafiken
\usepackage{pdfpages}
\usepackage{array}							%Tabellenextra
\usepackage{setspace}
\usepackage{url}

\usepackage[colorlinks=false, allcolors=black, urlbordercolor={1 1 1}, linkbordercolor={1 1 1}, pdftitle = {Konzeption und Entwicklung eines E-Learning Plattform für Usability},pdfauthor = {Juale Mercan}, linktocpage=true]{hyperref}
%\typearea[current]{calc}
%\addtolength{\skip\footins}{8mm}
%\addtolength{\footnotesep}{2mm}
\onehalfspacing
%\setlength{\parskip}{6pt}

%\usepackage[tocgraduated]{tocstyle}
%\usetocstyle{allwithdot} 


%\setlength{\bibhang}{1em} %Länge ggf. anpassen
%\setlength{\bibitemsep}{0.25\baselineskip} %Länge ggf. anpassen

\renewcommand*\chapterheadstartvskip{\vspace*{-1cm}}
\renewcommand*{\sectfont}{\bfseries}
\usepackage{geometry}
\geometry{includehead, includefoot, left=3.5cm,right=2.5cm,top=2cm,bottom=2cm}
\usepackage[headings]{fullpage}
\usepackage[format=default,font=footnotesize,labelfont=bf]{caption}

\begin{document}
\pagenumbering{roman}
\newcommand{\HRule}{\rule{\linewidth}{0.0mm}}
\thispagestyle{empty}
\begin{titlepage}
\begin{center}


% Upper part of the page
\textsc{\LARGE Bachelorthesis}\\[0.5cm]
%
%\HRule \\[0.4cm]
%\large {Zur Erlangung des akademischen Grades eines \linebreak Bachelors of Science \linebreak über das Thema}
%

% Title
\HRule \\[1.5 cm]
{ \LARGE \bfseries Konzeption und Entwicklung einer \linebreak interaktiven e-learning
Plattform \linebreak für Usability Inhalte im Kontext \linebreak betrieblicher
Umweltinfomationsysteme}\\[3.2 cm]%\linebreak \smallskip  

% {Eingereicht am Fachbereich 2 - Ingenieurwissenschaften II \newline der Hochschule für Technik und Wirtschaft Berlin}\\[2.1cm]


{von: Juale Mercan \linebreak \small Matrikel - Nr.: 0528812}\\[1.0cm]

% Author and supervisor
\begin{minipage}{0.4\textwidth}
\begin{flushleft} \normalsize
\emph{Erstbetreuer:}\\ Volker Wohlgemuth
\end{flushleft}
\end{minipage}
\begin{minipage}{0.4\textwidth}
\begin{flushright} \normalsize
\emph{Zweitbetreuer:} \\  

\end{flushright}
\end{minipage}
% Bottom of the page

\HRule \\[0.3cm]
{\normalsize Berlin, den \today}\linebreak
{\small (Tag der Einreichung)}
\HRule \\[0.3cm]
\begin{minipage}{0.2\textwidth}

\includegraphics[viewport=0 0 60 60]{HTW_Logo_4c.pdf}

\end{minipage}

\end{center}
\end{titlepage}
\thispagestyle{plain}

{\bf Angaben zur Person:}

		\setlength{\extrarowheight}{9pt}
		\begin{tabular}{ >{\em}b{4cm} >{}p{7cm}}
		Ersteller der Arbeit: & Juale Mercan\\
		Geburtsdatum: & 21.12.1984\\
		Geburtsort: & Razgrad, Bulgairen\\
		Anschrift: & Plesser Str. 12 \newline 12435 Berlin\\
		\end{tabular}\\\\
		
{\bf Angaben zur Hochschule:}

		\setlength{\extrarowheight}{9pt}
		\begin{tabular}{ >{\em}b{4cm} >{}p{7cm}}
		Hochschule: & HTW Berlin\\
		Anschrift: & Wilhelminenhofstraße 75A \newline 12459 Berlin\\
		Fachbereich: & Ingenieurwissenschaften II\\
		Studiengang: & Betriebliche Umweltinformatik\\
		Betreuer: & Prof. Dr. Volker Wohlgemuth\\
		\end{tabular}\\\\

% {\bf Angaben zum Unternehmen:}
% 
% 		\setlength{\extrarowheight}{9pt}
% 		\begin{tabular}{ >{\em}b{4cm} >{}p{7cm}}
% 		Firmenname & Volkswagen AG\\
% 		Bereich: & Forschung und Entwicklung\\
% 		Abteilung: & Umwelt Produktion\\
% 		Anschrift: & Brieffach 1897/0 \newline 38436 Wolfsburg\\
% 		Betreuer: & Roman Meininghaus \newline Steffen Witte\\
% 		\end{tabular}
	
\thispagestyle{plain}
\renewcommand{\abstractname}{Danksagung}
\begin{abstract}
Ein besonderer Dank gilt meinen Mitbewohnern und meiner Familie die mir während der arbeitsintensiven Zeit mit Rat und Tat zur Seite standen und ohne deren Unterstützung mein Studium und diese Arbeit nicht möglich gewesen wären. 

Danken möchte ich auch Prof. Dr. Wohlgemuth für die Vermittlung dieser Bachelorarbeit und das damit in mich gesetzte Vertrauen.
\end{abstract}
\renewcommand{\abstractname}{Zusammenfassung}
\begin{abstract}

\end{abstract}

\renewcommand{\abstractname}{Abstract}
\begin{abstract}
English bla bla
\end{abstract}

\tableofcontents
\listoffigures

\chapter{Einleitung}\label{Einleitung}
\pagenumbering{arabic}
\section{Motivation, Problemstellung}
Lebenslanges Lernen dank spielerisch gestalteten E-learning tools. Usability Wissen soll so interessant aufbereitet werden, dass es spielerisch aufgenommen werden kann. Das Lernen und verinnerlichen von Usability bei der Entwicklung von Software soll innerbetrieblich gepflegt werden können. 
Der Content sollte leicht und intuitiv aufbereitet sein. Für die vielseitige Verwendbarkeit. Spielerisch und Sozial vermitteln was relevant ist für die Entwicklung von Benutzerfreundlichen Anwendungen und Seiten.



\begin{itemize}

\item Wie können Inhalte (Usability Glossar)Vermittelt werden
\item Was soll vermittelt werden? Aufbereitung des Content als Lückentext, Bildaufgabe, T
\item Wer sind die Zielpersonen Benutzer?
\item Warum ist es Sinnvoll E-Learning Tools für BUIS entwickler anzubieten?
\item BUIS sind nicht intuitiv und selten eingebungen, folgen, werden ungern genutzt
\item Ein E-learning Tool welches spielerisch in Leerlaufphasen genutzt werden kann
\item E-learning Plattform als Autorenprogramm um Inhalte auf BUIS anpassen zu können
\item Es gibt kein Know How bei den KMU entwicklern
\item keine Ansprechend aufbereiteten Inhalte um sich Usability anzueignen
\item Was gibt es auf dem Markt und warum ist das Thema noch nicht angenommen
\item Angebote die verfügbar sind noch nicht angenommen und wie müsste das Spiel aufgebaut sein damit sich KMU Entwickler damit beschäftigen
\item Innerhalb des KOMET Projektes gibt es bereits folgenden Prototypen und 
\end{itemize}

\section{Zielsetzung}
Beschreibung der Punkte die ich erreichen möchte:
Beschreibung des Teils den ich ausarbeiten möchte
Ums

\section{Ausfbau der Arbeit}
3 Konzepte für das Quiz, implementieren -> Testen ->
Lernmodus das Glossar durch skipen Bilder integrieren 
nicht nur sequenzielles lernen sondern auch  
Quiz  Wiederholungen zählen, 
Lückentext 


\chapter{Theoretische Grundlagen BUIS}
\section{Definition und Kategorisierung}
Morphologischer Kasten + Was alles zu BUIS gehört
\section{Einsatzgebiete und Hemmschwellen bei der Nutzung von BUIS}
Ursprung der Probleme bei der Nutzung ist oft die geringe Beachtung von Usability Richtlinien und Werten bei der Entwicklung der BUIS. 
BUIS verfolgen einen ganzheitlichen Ansatz werden dennoch meist punktuell und für die konkreten Bedürfnisse in KMUS entwickelt. 
Am Beispiel von Umsys bzw. Umberto ist erkennbar wie wenig Beachtung dem Nutzer und der Zielgruppe bei der Entwicklung des E-Learning Tools geschenkt wurde. 

Probleme bei den Kategorien und alles kann sich auf die Konzeption und Entwicklung zurück führen lassen.
%Buis können gestärkt werden durch erleichtere und bessere Anwendbarkeit
Das sind die Probleme
\section{E-learning Tools und oder Tutorials der BUIS}
\section{Usability von BUIS}

\section{Lösungsansatz Gamification von E-learning, Usability}
Etwickler haben zu wenig Ahnung von Usability und sollen diese Inhalte mit dem zu entwickelnden Tool verinnerlichen. 

\chapter{Theoretische Grundlagen E-Learning und Usabiliy}
\section{E-learning im betrieblichen Kontext}
Einsatzmöglichkeiten und häufig genutze Tools in der Betrieblichen Umweltinformatik 
Wie werden Entwickler weiter geschult?
Wer ist meine Zielgruppe?  Ent
Was soll das Ergebnis des Kurses sein?

\section{E-learning und Gamification}
Für das Lernen im 21. Jahrhundert setzt sich der Konstruktivismus als Lernthorie durch. Das bedeutet, Lernen im Konstruktivismmus wird verstanden als die Konstruktion von Wissen auf der Basis individuellen Vorwissens; -> daher muss immer auf den einzelnen Lehrenden Eingegangen werden. \cite[S.8]{1}
\section{Methoden des E-learning}
\section{Quiz als Methode}
\section{Lückentext als Methode}
\

\chapter{Konzeption}
\section{Rahmenbedingungen}
Innerhalb des KOMET Projektes soll ein E-Learning tool für die 128 Usability Begriffe und Definitionen als Glossar sind gegeben. \\Die Entwicklungsumgebung ist gegeben:
\begin{itemize}
\item{Webbasiertes Tool}
\item{HLMT/ HTML 5}
\item{Jason/ Java Skript}
\item{} 
\end{itemize}

Ein Quiz nach dem Modell von Quizduell, bei dem die Zeit und die Folge von richtigen Antworten aufgezeichnet werden und in den Highscore einfließen.

\section{Vorgehensweise}
Jetzt entwerfe ich ein E-learning tool was Entwicklern das beachten von
Usebility bei der Entwicklung von BUIS erleichtert. 

\section{Methoden}
\subsection{Glossar sequentiell}
\subsection{Glossar shuffle}
\subsection{Begriff + Definition + Bild}
\subsection{Begriff + Definition + TTS Button}
\subsection{Quiz}
\subsection{Lückentext (mit 4 Optionen)}
\subsection{Lückentext zum selbertippen?}

\section{} 
\chapter{Praxisteil}
\section{Prototyp und Evaluation der Methoden?}
Beispielbilder zu jedem Begriff ? Definition vorlesene lassen Text to Speech integrieren?
\section{}
\section{Glossar zum Spiel ausbauen}
In Anlehnung an das Spiel "Time Up"

\printbibliography
\end{document}